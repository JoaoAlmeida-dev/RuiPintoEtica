\paragraph{What is a Whistleblower?}
\textit{persons who report (within the organisation concerned or to an outside authority) or disclose (to the public) information on a wrongdoing obtained in a work-related context, help preventing damage and detecting threat or harm to the public interest that may otherwise remain hidden.}
\footnote{\url{https://ec.europa.eu/info/aid-development-cooperation-fundamental-rights/your-rights-eu/whistleblowers-protection_en}}


%\url{https://en.wikipedia.org/wiki/Whistleblower#:~:text=A%20whistleblower%20(also%20written%20as,%2C%20illicit%2C%20unsafe%20or%20fraudulent.}

%It is according to the european whistleblowers directive that a "whistleblower" is a person that \textit{"By reporting breaches of Union law that are harmful to the public interest, such persons act as ‘whistleblowers’"}
%\footnote{DIRECTIVE (EU) 2019/1937 OF THE EUROPEAN PARLIAMENT AND OF THE COUNCIL of 23 October 2019 on the protection of persons who report breaches of Union law \url{https://eur-lex.europa.eu/legal-content/EN/TXT/PDF/?uri=CELEX:32019L1937}}

\subsection{Primary Ethical questions}
During the whole process, Rui had no professional obligation to abide by, except the moral rules and laws that every citizen is placed under.

By accessing the internal servers/services of Sporting Football Club and obtaining confidential documents, even if done with a good intention in mind, is still breaking the privacy of the referenced Club and all the people that these documents mentioned.
Beyond being a privacy breach, we have to question ourselves on how did Rui Pinto actually gain access to these documents, he was not an employee of Sporting, he himself did not have personal access to the system, so he had to, at one point or another, breach the security of their system.
By persuading and promising his silence to the Big Football clubs regarding their confidential documents in exchange of money he committed a very severe crime of extorsion, manipulating the organizations to do what he wanted, holding these documents as a way of making them kneel before his will.
Here we have in question the main goal of Rui Pinto, was it to actually create a better Football scene and environment like the website auto-proclaimed or is it for self benefit?


\begin{itemize}
    \item Is it Ethical to "hack" into someone's email, essentially break their privacy, to know if that someone is involved in some sort of bad deed, some sort of corruption or law breaking?
    \item Is it right to break a law in order to find who is breaking the law?
    \item On the other hand, should privacy laws protect us so much that we have the freedom to hide unlawful acts behind it?
    \item Should Rui Pinto be classified as a Whistleblower?
    \item Should Rui be protected by the EU whistleblowers directive
\end{itemize}

\subsection{Ethical knowledge and sensibility, harmed entities}
\subsubsection{Precedents and analogies}
Rui Pinto, at a first glance of the facts does in fact seem like a person who is invested in doing the right thing, by publishing so much information about corrupt activities, which lead to the investigation of multiple parties, he effectively assisted the police in several other cases.
But things are not so simple with Rui as even before the case of Football Leaks, the cyber attacks to the \textit{Caledonian Bank} which he was involved in served as a precedent for his extortions of Doyen Sports, FC Sporting  Football Leaks.

%So when Rui Pinto made a new post on Football Leaks exposing their illicit actions, instead of calling the correspondent authorities he did not act according to this directive.

%When he was put in prison and extradited from Hungary back to Portugal, Pinto's life was put on hold.
\subsubsection{Who was harmed and why}
%they where 
\begin{itemize}
    \item The leaked football clubs and personalities are now at risk of being put under investigation.
    \item 
\end{itemize}

\subsubsection{Consider all the involved parties}
\begin{itemize}
    \item Rui Pinto, as he stated on the website, he was only trying to bring to light the corruption inside the Football Scene by exposing these corporations, but after taking into account his extorsion attempt to Doyen it is clear that he was also very much motivated by money and 
    \item Anibal Pinto was, at first, only protecting his client, but he is also suspect of having collaborated with the primary suspect RUi Pinto in his extorsion to Doyen.
    \item Doyen simply protected itself as it came into contact with the proposals from Rui Pinto, declaring him to the authorities.
    \item The organizations that had their documents leaked, will probably have their schedules altered due to possible investigations.
\end{itemize}

\subsection{Systematic analysis - Step 6}

\subsubsection{Professional Standarts}

\subsubsection{Roles and Responsibilities}

\begin{itemize}
    \item Doyen Sports, as the primary entity that contacted the authorities regarding Rui Pinto
    \item Rui Pinto as the suspect of committing many crimes and responsible for starting many investigations around various personalities on the Football scene.
    \item SC Sporting and other football clubs and personalities as the "victims", the ones that were 
    
\end{itemize}

\subsubsection{Stakeholders}

\paragraph{Benefited stakeholders}


\paragraph{Harmed stakeholders}
    \begin{enumerate}
        \item From all the people involved in this case, Rui Pinto is the most important one, he is the one who claimed ownership of the popular website, he is the one who is being hold on trial directly.
        \item All of the anonymous people helping in the management of the Leaks are as important as Pinto himself, the only difference is that as we cannot identify these people, the case and all of the social impact is directed at Rui Pinto.
        \item His lawyer Anibal Pinto is also an important "piece of the puzzle", without him, pinto would not have been able to contest on his first attack to \textit{Caledonian Bank}.
        \item Doyen Sports, who was extorted by Pinto and his lawyer, and had its agreements with FC Twente published online which led to the clubs ban from European Football for three years.
        \item The Portuguese, French, Belgium and Dutch Government
    \end{enumerate}
\subsubsection{Politics}

\subsubsection{Normative Ethics}

\paragraph{Utilitarism}
    is the ethical way of thinking where an action is deemed ethical when it tends to benefit all parties involved and non ethical when it tends to harm the parties.
    This action, also must be impartial and numb to race, gender, and an other form of judgement, each person amounts to the same "weight".

    Rui Pinto before deciding to create Football Leaks had to balance his actions on a scale of cost versus benefit
    \begin{enumerate}
        \item [Benefit] - The discovery of several corruption cases.
        \item [Cost] - Disrespect the right to privacy of all the affected companies and personalities, potentially breaking this right with no outcome.
        If someone that had not practiced any wrongdoing had their privacy breached by Pinto, they were a victim and they only suffered, giving no "benefit" to Pinto.
    \end{enumerate}

\paragraph{Virtue Ethics}
    focusses itself on the will of the actor, the ulterior motives and his moral character 
    
    Rui Pinto's actions always had two sides associated with them.
    \begin{enumerate}
        \item An inherently good side where Pinto's goals were to eradicate the corruption in the Football scene.
        \item A more selfish side where he tries to obtain money through unethical ways such as extorsion, holding his information as a kind of "hostage".
    \end{enumerate}

\paragraph{Deontologic Ethics}

\subsection{Conclusions}



    