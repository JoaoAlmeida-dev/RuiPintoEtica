\paragraph{Disclamer} For the purposes of this analysis, I am taking into account only what is described in the above section, and considering that all the accusations against Pinto are true

\paragraph{What is a Whistleblower?}
\textit{people who report (within the organization concerned or to an outside authority) or disclose (to the public) information on a wrongdoing obtained in a work-related context, help preventing damage and detecting threat or harm to the public interest that may otherwise remain hidden.}
\footnote{[16-05-2022 22:40] \url{https://ec.europa.eu/info/aid-development-cooperation-fundamental-rights/your-rights-eu/whistleblowers-protection_en}}


%\url{https://en.wikipedia.org/wiki/Whistleblower#:~:text=A%20whistleblower%20(also%20written%20as,%2C%20illicit%2C%20unsafe%20or%20fraudulent.}

%It is according to the european whistleblowers directive that a "whistleblower" is a person that \textit{"By reporting breaches of Union law that are harmful to the public interest, such persons act as ‘whistleblowers’"}
%\footnote{DIRECTIVE (EU) 2019/1937 OF THE EUROPEAN PARLIAMENT AND OF THE COUNCIL of 23 October 2019 on the protection of persons who report breaches of Union law \url{https://eur-lex.europa.eu/legal-content/EN/TXT/PDF/?uri=CELEX:32019L1937}}

\subsection{Primary Ethical questions}

By publishing several confidential documents of Sporting Football Club, even if done with a good intention in mind, he still broke the privacy of the referenced Club and all the people that were mentioned in these documents.
Beyond being a privacy breach, we have to question ourselves on how did Rui Pinto actually gain access to these documents.

By persuading and promising his silence to Doyen sports regarding their confidential documents in exchange of money he committed a very severe crime of extorsion, manipulating the organizations to do what he wanted, holding these documents as a way of making them kneel before his will.
Here we have in question the main goal of Rui Pinto, was it to actually create a better Football scene and environment like the website auto-proclaimed or is it for self benefit?


\begin{itemize}
    \item Is it Ethical to break someone's privacy, based only on suspicion that the person is involved in some sort of bad deed, corruption or law breaking?
    \item On the other hand, should privacy laws protect us so much that we have the freedom to hide unlawful acts behind it?
    \item Should Rui Pinto be classified as a Whistleblower and therefore be protected by the EU whistleblowers directive?
\end{itemize}

\subsection{Ethical knowledge and sensibility, harmed entities}
\subsubsection{Precedents and analogies}
Rui Pinto, at a first glance of the facts does in fact seem like a person who is invested in doing the right thing, by publishing so much information about corrupt activities, which lead to the investigation of multiple parties, he effectively assisted the police in several other cases.
But things are not so simple with Rui as even before the case of Football Leaks, the cyber attacks to the \textit{Caledonian Bank} which he was involved in served as a precedent for his extortions of Doyen Sports, FC Sporting  Football Leaks.

\subsection{Systematic analysis - Step 6}

\subsubsection{Professional Standarts}
Since Rui Pinto had a respectable amount of knowledge on computer technology, he can be somewhat considered a Software Engineering professional, therefore we can have him accountable for not following the IEEE Software Engineering Code of Ethics Professional Practice.
\footnote{[22-05-2022 23:23]\url{https://www.researchgate.net/publication/278417404_Software_Engineering_Code_of_Ethics_and_Professional_Practice}}

In the described case Pinto violates three codes of conduct.
\begin{itemize}
    \item 1.03 for approving software that risks privacy.
    \item 1.04 for not contacting the respective authorities upon having suspicion of the wrong doings practiced in the mentioned organizations.
    \item 4.03 for engaging in extorsion, an improper financial practice.
\end{itemize}



\subsubsection{Roles and Responsibilities}
    Doyen Sports served as the primary source of accusation towards Rui Pinto, the suspect of committing multiple crimes due to the managing of the Football Leaks website and responsible for starting several investigations on big organizations and personalities of the Football Scene which ended up as the "victims" of the case, several of these had their privacy broken into by Rui Pinto and confidential information leaked into the public through the Leaks website.

\subsubsection{Stakeholders}

Rui Pinto is very fortunate for having his lawyer, without him , he would not have been able to contest on his first attack to \textit{Caledonian Bank}, besides this they \textbf{benefited} by having Pinto's view of the football scene somewhat realized when operating the website.
Although Doyen Sports and all of the leaked personalities where \textbf{harmed}, as they had their privacy broken into, the Portuguese, French, Belgium and other Governments lightly benefited from the whole case, due to starting based on the leaks from the events mentioned. 
Some of these Governments with the cooperation of Rui Pinto himself.

%\paragraph{Benefited stakeholders}
%    Rui Pinto was benefited by having his personal will fulfilled when operating the website.
%
%\paragraph{Harmed stakeholders}
%    \begin{enumerate}
%        \item From all the people involved in this case, Rui Pinto is the most important one, he is the one being held on trial directly.
%        \item Whiteout his lawyer, Anibal Pinto, Pinto would not have been able to contest on his first attack to \textit{Caledonian Bank}.
%        \item All of the anonymous people helping in the management of the Leaks are as important as Pinto himself.
%        \item Doyen Sports, who was extorted by Pinto and his lawyer
%        \item The Portuguese, French, Belgium and Dutch Government
%    \end{enumerate}
%
%\paragraph{Who was harmed and why}
%\begin{itemize}
%    \item The leaked football clubs and personalities are now at risk of being put under investigation.
%    \item These personalities had their privacy right disrespected, after the event private information like their salary is now public knowledge.
%    \item 
%\end{itemize}

\paragraph{Consider all the involved parties}
While it is true that Anibal Pinto was only protecting his client, Rui Pinto, he is a suspect of helping Rui in his extorsion acts against Doyen, which only protected itself by accusing Pinto to the authorities.
His client, on the surface, seems to only care about the legitimacy of the Football Scene, his extorsion attempt contradicts this purist train of thought, if his true goals where only the removal of corruption, then he would contact the authorities instead of "taking matters into his own hands".

%\begin{itemize}
%    \item Rui Pinto, as he stated on the website, was only trying to bring to light the corruption inside the Football Scene by exposing these corporations, but after taking into account his extorsion attempt to Doyen it is clear that he was also very much motivated by money and 
%    \item Anibal Pinto was, at first, only protecting his client, but he is also suspect of having collaborated with the primary suspect RUi Pinto in his extorsion to Doyen.
%    \item Doyen simply protected itself as it came into contact with the proposals from Rui Pinto, declaring him to the authorities.
%    \item The organizations that had their documents leaked, will probably have their schedules altered due to possible investigations.
%\end{itemize}


\subsubsection{Politics}
\paragraph{}
The only legislation that somewhat applies to this case is the European normative of whistleblowers
\footnote{\url{https://ec.europa.eu/info/aid-development-cooperation-fundamental-rights/your-rights-eu/whistleblowers-protection_en}}


\subsubsection{Normative Ethics}

\paragraph{Utilitarism}
    is the ethical way of thinking where an action is deemed ethical when it tends to benefit all parties involved and non ethical when it tends to harm the parties.
    This action, must also be impartial and numb to race, gender, and an other form of judgement, each person amounts to the same "weight".

    Rui Pinto before deciding to create Football Leaks had to balance his actions on a scale of cost versus benefit
    \begin{enumerate}
        \item [Benefit] - The discovery of several corruption cases.
        \item [Cost] - Disrespect the right to privacy of all the affected companies and personalities, potentially breaking this right with no outcome.
    \end{enumerate}

    Taking this balance into account and the Utilitarian Ethics, the actions of Rui Pinto would be ethical, due to the high benefit to the society in exposing various cases of corruption.
    Since the benefits outweigh the costs this actions had a good impact in the general society's happiness and so would be considered ethical.

\paragraph{Kant's Deontologic Ethics}
    differs from Utilitarianism by stating that some actions are immoral even if they achieve global happiness and pleasure.
    
    Kant believes that our human emotions and biases should not be involved in any  moral action.
    Our morality should serve as a framework of rational rules that dictate what actions to take and what not to.
    
    Kant describes several Categorical Imperatives, from which every moral norm can be derived:
    \begin{itemize}
        \item \textit{“Act only according to that maxim whereby you can at the same time will that it should become a universal law without contradiction.”}
        \item \textit{“Act in such a way that you treat humanity, whether in your own person or in the person of any other, never merely as a means to an end but always at the same time as an end.” }
        \item \textit{“Therefore, every rational being must so act as if he were through his maxim always a legislating member in the universal kingdom of ends.”}
    \end{itemize}
    
    For an action to be considered ethical by Kant, it's outcome does not matter, the action itself must be driven by the actor's obligation, must be well thought out and abide by his morality rules.
    \footnote{\url{https://sevenpillarsinstitute.org/kantian-duty-based-deontological-ethics/}}

    In the case of whistleblowers, specifically Rui Pinto's case, for his actions to be ethical they must remain ethical in the hypothetical scenario where everyone can invade everyone else's privacy with the justification of finding if the invaded is doing any illicit activity.
    
    This is not feasible, if the valid justification for breaching someone else's privacy is simply suspicion then the privacy rights are never respected and end up losing their value.

    
\subsection{Conclusions}
    \subsubsection{Key Ethical Questions}

    \paragraph{Is it Ethical to break someone's privacy, based only on suspicion that the person is involved in some sort of bad deed, corruption or law breaking?}
    We can set up a small analogy for analyzing this question.
    An officer needs to go up the hierarchy of justice, demonstrating why he is justified and has suspicions of someone, therefore it would be beneficial to invade his house and "catch" the suspect in his comfortable habitat.
    
    A whistleblower is very similar to a police officer that does not need a search warrant to enter a house.
    If one does not need a very well justified reason to invade someone's privacy, he can use his suspicions as an excuse to break into a home.
    
    Therefore, it is \textbf{not} ethical to break someone's privacy based solely on suspicion.

    \paragraph{Should privacy laws protect us so much that we have the freedom to hide unlawful acts behind it?}
    This is mainly a question of "trading-off" privacy with better law enforcement.
    The more the authorities control what individuals and companies do, the more they can catch illicit actions but the less privacy that person or company has.
    While there needs to exist some sort of supervision on individual activities, this supervision cannot be so mild that everyone can disrespect the law, using their privacy to shield themselves.

    A possible solution to a question like this could be a system that would allow the authorities to analyze the internals of an organization with a certain level of anonymity.
    That way the aspects that regard the law are shown and the aspects that regard the individual's privacy remain in secret.
    Although it is an interesting idea, a system like this is very difficult to make, as it needs to be highly sophisticated and highly specific for each law and each different organization.
    It also needs to counteract the problem where a large amount of anonymous data can be compiled to create non anonymous data, essentially reverting the anonymity of the information.
    
    
    
    \paragraph{Should Rui Pinto be classified as a Whistleblower and therefore be protected by the EU whistleblowers directive?}
    According to the definition of a whistleblower in the European whistleblowers directive, Rui Pinto cannot be considered a whistleblower, this is because he is not an employee in the companies referenced in the case, therefore he did not access the confidential documents referenced in a legal way.
    Therefore, Rui Pinto is not a whistleblower and should not be protected by the directive.

    \subsubsection{Lessons for the future}
    If there is a suspicion that someone or some organization is committing illegal acts, this suspicion should be expressed to the corresponding authorities.
    Rui Pinto should have contacted the authorities as he is not a working member of those organizations and therefore could not contact the internal chain of command to report his suspicions.
    Then the authorities would decide on how they would proceed with the case and if deemed necessary, in the future they could justify breaking the privacy rights of the suspect organizations, but until then, that right would be respected and remain unbroken.
    
    The act of whistleblowing, in the way that Pinto executed it is not ethical at all, the right to privacy cannot be broken on a mere suspicion, only in the case where the authorities are presented with multiple valid reasons to investigate someone is when their privacy can be broken into.

