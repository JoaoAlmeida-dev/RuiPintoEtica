\paragraph{What is a Whistleblower?} A person who reveals information regarding illegal or immoral activities performed by a certain organization.
Whistleblowers can use internal or external channels to communicate their information, often putting their safety at risk, due to the possible retaliations that the organization in cause can apply, for example: if the Whistleblower is an employee at the organization their contract often gets immediately terminated.


%\url{https://en.wikipedia.org/wiki/Whistleblower#:~:text=A%20whistleblower%20(also%20written%20as,%2C%20illicit%2C%20unsafe%20or%20fraudulent.}

\subsection{Primary Ethical questions}
During the whole process, Rui had no professional obligation to abide by, except the moral rules and laws that every citizen is placed under.

By accessing the internal servers/services of Sporting Football Club and obtaining confidential documents, even if done with a good intention in mind, is still breaking the privacy of the referenced Club and all the people that these documents mentioned.
Beyond being a privacy breach, we have to question ourselves on how did Rui Pinto actually gain access to these documents, he was not an employee of Sporting, he himself did not have personal access to the system, so he had to, at one point or another, breach the security  of their system.
By persuading and promising his silence to the Big Football clubs regarding their confidential documents in exchange of money he committed a very severe crime of extorsion, manipulating the organizations to do what he wanted, holding these documents as a way of making them kneel before his will.
Here we have in question the main goal of Rui Pinto, was it to actually create a better Football scene and environment like the website auto-proclaimed or is it for self benefit?

Is it Ethical to "hack" into someone's email, essentially break their privacy, to know if that someone is involved in some sort of bad deed, some sort of corruption or law breaking?
Is it right to break a law in order to find who is breaking the law?
On the other hand, should privacy laws protect us so much that we have the freedom to hide unlawful acts behind it?


\subsection{Ethical knowledge and sensibility, harmed entities}
Rui Pinto, at a first glance of the facts does not seem like a super righteous person, even before the case of Football Leaks he was involved in cyber attacks to the \textit{Caledonian Bank}, allegedly trying to transfer a big sum of money, possibly to be used for his own personal needs, serving as a precedent for his extortions surrounding Football Leaks, we can conclude that the betterment of the Football scene was not his only goal.
Which again comes to the limelight when the people affected are referenced, countless careers and lives were impacted by the publications of the website.

\subsection{Systematic analysis}
\subsubsection{Professional Standarts}
\subsubsection{Involved Actors}
The involved actors are primarily Rui Pinto and his anonymous team on the Football Leaks website, his lawyer, and on the other side of the stick there is all the impacted organizations, such as FC Sporting, its team coach, Jorge Jesus, Cristiano Ronaldo, which was involved in a leaked case on the website.
\subsubsection{Politics and legislation}
\subsubsection{Normative theories of Ethics}


